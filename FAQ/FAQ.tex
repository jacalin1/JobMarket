\input{.econtexRoot}\input{\econtexRoot/.econtexPaths}\documentclass{\classes/econtex}
\usepackage{\packages/JobMarket}
\usepackage{\packages/econtexSetup}
% Allow links to items inside itemize environment
% https://tex.stackexchange.com/questions/81949/can-individual-items-in-an-itemize-list-be-labelled-and-hyperref-linked/81952#81952
% \AtBeginEnvironment{itemize}{\apptocmd{\item}{\phantomsection}{}{\errmessage{couldn't patch item}}}
\renewcommand{\thepage}{} % Get rid of page numbers, which don't convert to md or html

\providecommand\phantomsection{}
\provideboolean{MyNotes}\setboolean{MyNotes}{false} % Whether to show marginalia

\pagestyle{plain}
\hypersetup{pageanchor=false} % prevents useless warning messages
\begin{document}
\hfill{\tiny \jobname, \today} \vspace{.1in}

\begin{verbatimwrite}{\jobname.title}
  Frequently Asked Questions About the Job Market Process
\end{verbatimwrite}

\centerline{\Large Frequently Asked Questions About the Job Market Process}\medskip\medskip

\centerline{\today}\medskip\medskip

\ifdvi\large\fi

% \ifthenelse{\boolean{MyNotes}}{\marginpar{\tiny }}{}
\begin{enumerate}

\item Who is in charge of the job market process?
  \begin{quote}
    You are.  You are responsible for your own job market search.  (The economics PhD job market is a very capitalist institution).

    The \JMPO is the overall coordinator of the process, responsible (for example) for updating these notes.  But you should \textit{not} wait for specific reminders and prodding by the JMPO before doing the things you are supposed to do. If you fail to do things that are in your own best interest, and which nobody will do for you, there is nobody to blame but yourself.

  \end{quote}

\item  Who is eligible to go on the market?
  \begin{quote}  
    Students who have the approval of their advisors (you need to make
    sure that you consult with ALL faculty members whom you will ask to
    write letters, not just your primary advisor -- all the advisors
    should be encouraged to come to your practice job market talk); and
    who either have not participated in the job market process before,
    or participated but did not find a job (this is meant to reassure
    you, not to worry you; we have a very good record of students
    ultimately getting jobs).

    \ifdvi\phantomsection\hypertarget{DecideByWhen}{DecideByWhen}\fi
    
  \end{quote}

\item  When do I need to make a decision about whether I am going on
  the market?

  \begin{quote}
    If you are a 6th year student this year, you don't have a choice: If
    you have an acceptable job market paper in time, you must go on the
    market; if you don't have an acceptable job market paper in time, you
    are in deep trouble, since the department must approve your job market
    paper, you can't go on the market without a paper; and you can't stay
    in the program for a 7th year.  (See the graduate \href{http://www.econ2.jhu.edu/pdf/Econ\_Grad\_Handbook.pdf}{handbook} for a fuller discussion of these issues.) 

      For other students (5th, or in very special cases maybe 4th year students), basically you should be talking to your advisors (ALL of them, not just your main advisor) to make a final decision by the end of the first week of October.  It is costly to you and to us for you to withdraw after the job candidate website `goes public' in mid-October.  It is even costlier to withdraw after you send out applications to employers in early-to-mid November.  Anyone who attempts to withdraw after mid-December will be shot.

      \begin{comment}

        In the dept meeting on 2006-10-17 a question was raised about
        whether we are really serious about our policy, as stated by Larry,
        that ``there is no such thing as a 7th year student.''  

        The answer is yes.

        There was then some discussion about the relationship between going
        on the job market and remaining in the program, caused by the fact
        that there are currently three students (Tereanu, Svitil, Lu) who
        are clearly not ready for the job market but have not given up all
        hope of finishing a dissertation by August of next year.

        The resolution (subsequently clarified in emails on 2006-10-18 and
        2006-10-19 and an email to Zhou Lu on 2006-10-19 from CDC) was as
        follows.

        Our declared public policy (see the graduate \href{handbook}{http://www.econ2.jhu.edu/pdf/Econ_Grad_Handbook.pdf}) is that nobody can enroll for a 7th year
        in the program.  In practice, if in the opinion of an advisor and
        second advisor, a student's dissertation is essentially complete as
        of August of the 6th year, we may relax this rigor slightly in order
        to arrange a defense and final ``cleaning up'' of details - but \textit{  only} in the case where the dissertation is in all essentials
        complete.

        So, an important real and psychological point here is that failure
        to be included in the official list of candidates is almost (but not
        quite) equivalent to the end of all hope of obtaining a JHU PhD.  If
        a student uses wisely the time freed up by not participating in the
        job market to conduct an intense burst of high quality research, it
        is in principle quite possible that they will finish a dissertation
        despite their prior lack of progress.  The exclusion from the job
        market process may concentrate their minds in a way that has not
        happened before.

        Furthermore, if their dissertation defense occurs before the
        deadline for the `on-the-market' decision in the following year (mid
        October), the student can apply to jobs themselves (with the usual
        process, including advisors writing letters and giving them to the
        \JMStaff, etc), but without their name being officially
        included in list of `on-the-market' students.  They can also have a page on the dept
        website to which they can refer employers, but will not be included
        in the list of students ``on the market'' clickable from the main
        dept web page.  The
        situation will be explained candidly to employers as resulting from
        our inflexible rule that students must be on the market in their 6th
        year or earlier.  This is what we did with Farhan Hameed this year,
        and he did eventually get a couple of mediocre job offers, one of
        which he has accepted.

      \end{comment}


    \end{quote}
  \item  If I'm not sure whether I'll be on the market or not,
    which steps should I take (e.g.\ should I schedule a job 
    seminar)?

    \begin{quote}
      Unless you are pretty sure you are \textit{not} going to be on the
      market, you should behave as though you are sure you \textit{are}.  So,
      yes, go ahead and schedule a practice job talk, etc.

    \end{quote}
  \item  Can I `partly' go on the job market?  That is, apply to a few
    selected employers (say, the World Bank's YP program) only, and 
    if that doesn't work go on the market more fully next year?

    \begin{quote}
      No.  This does not work.  The things you (and your advisors) need to do in order for you to
      get a job are basically the same: Get letters of recommendation, 
      write a respectable job market paper, etc.  If you do not do them
      properly, you won't get a job anywhere; and, if you do them well,
      you will have good options.  A half-baked application is worse than
      none at all, because you make a bad impression on that employer, and 
      employers talk to each other a lot, so that bad impression might damage
      your chances elsewhere in the future as well.  

      You can't go on the job market twice (with support from JHU, anyway; and, you won't
      get a job without support from JHU).

    \end{quote}
  \item  Can I use a coauthored paper as my job market paper?

    \begin{quote}
    Yes - if the coauthor is another student, and that student is not
    also using the same paper.  No, if the paper is with one of your
    advisors.  Maybe, if the paper is with someone who is not a student
    and not an advisor (and not famous).  You will need to discuss with your advisor.
    \hypertarget{Where-Should-I-Look-for-Jobs}{}

  \end{quote}

\item  Where should I look for jobs? 

  \begin{quote}
    There is one main source: \JOE

      No other source that I know of is large enough to direct everyone to. (If any student finds a rich place that everyone should be looking at for job postings, please let me know).

      \hypertarget{Region-Specific-Job-Markets}{}
      
      There are some region-specific job markets that may be worth paying attention to if you are particularly interested in a job in the region in question.  For example, over the years there have been several efforts to produce a European job market patterned on the US market.  Recently (I think starting in 2017) a new effort by the \href{https://www.eeassoc.org/}{European Economic Association} has been launched, which seems to have learned some lessons from past efforts. Some European institutions (universities; central banks) have said they have switched their recruiting efforts to the European from the American market.  Some institutions are treating participation in this market as a screening device: Anyone at the ASSA meetings can say they'd be delighted to have a job in Europe when a European institution interviews them, but then when they have offers from comparably ranked American institutions they tend to go to the American employer.  Flying all the way to Europe is a costly and time-consuming thing to do, and for PhD candidates both money and time are at a premium. So if you do participate in the European market, that does significantly increase your odds of getting a job offer from a European institution, compared to your odds if you were to interview with the same institution in the US. A potential strategy here is to apply to jobs at European institutions and see if you get invited to an interview.  If you have no interview requests at institutions you might be willing to take a job from, you can cancel any plans you have made to go to Europe.  Alternatively, if your job market prospects in the US market are not looking very favorable, it might be worth sending a costly signal to European institutions by going to the European market.

      Similar logic probably applies to other regional or country-specifc markets.
    \end{quote}
    
    \item What is the timing of posting of jobs on JOE? 

      \begin{quote}
        Each year is slightly different from the previous. See the table (credit: Daniel Garcia) 
        that shows the number of new jobs posted on JOE between different dates 
        during 2016's job cycle:
        \ifdvi\phantomsection\hypertarget{JobPostDates}{(JobPostDates)}\fi
        \begin{table}[h]
          \centering
          \begin{tabular}{|l|c|}
            \hline
            Dates                   & \# of job postings on JOE \\ \hline
            Aug 1 -- Oct 11   & 740                       \\ \hline
            Oct 11 -- Nov 1 & 340                       \\ \hline
            Nov 1 -- Dec 1 & 290                       \\ \hline
            Dec 1 -- Jan 1  & 115                       \\ \hline
          \end{tabular}
        \end{table}
      \end{quote}

  \item How many references should I have?

    \begin{quote}
      Absolute minimum (and ideal number) of 3.  Absolute maxmimum of 5 (under
      extremely special circumstances).  

      You should list as references \textit{only} people who have \textit{agreed}
      to write letters of recommendation for you.  They should know \textit{long} in advance that you will be asking them for a letter.  5 is the absolute
      maximum, and is NOT preferred - employers will not read 5 letters,
      except in rare cases.  3 is the optimal number.  4 is acceptable, and
      better than 3 if you really have 4 people who are very familiar with
      your work \textit{or} your qualifications, or if there is someone who can
      uniquely testify to a particular talent (brilliant work as a Fed RA,
      for example) that nobody else can see.  5 is reserved for cases where
      you have many people both inside the dept and outside who are
      thoroughly familiar with some aspect of your work (though it doesn't
      have to be your thesis; someone for whom you worked as an RA might
      write an excellent letter about how you performed).  You should \textit{not} have 5 references if some of those people are only vaguely
      acquainted with what you have done.  You should probably have at least
      two Hopkins faculty members.

      Remember that someone reviewing your file is not overcome with
      a thrill of anticipation when they see that you have 4 or 5 letters; they
      are overcome with a pang of annoyance at the extra work.  So don't foolishly
      imagine that ``more is always better;'' include the extra letters if
      there's a good reason, but not otherwise.


    \end{quote}
  \item  Can I send different sets of letters to different employers?

    \begin{quote}
      If every student had different letters for every employer, it would be impossible for {\JMStaffName} to get it all correct (and a huge amount of work even to try).

      {\JMStaffName} is prepared to handle different sets of letters to different categories of employers (usually, academic vs nonacademic) but you need to explicitly label those categories for her, and make SURE your letter writers label the letters clearly in the different categories when they send them to her.  Like, maybe, letters for academic institutions should have \texttt{academi} in the name of the PDF.

      Depending on {\JMStaffName}'s bandwidth, you might ask for a \textit{few} special-case deviations from your usual set of letters.

      If the motivation is that you have a potential letter writer who has particular contacts at particular institutions that might be helpful, another (perhaps better) option is to have that writer email their contact at the institution directly, and send either a full letter or a personal email endorsement (or both).  This is more work for the letter writer, but is likely to be more effective than just having their letter included as (say) a fourth or, god forbid, fifth letter.

    \end{quote}
  \item  How is the market for economists this year?

    \begin{quote}
      We never really know until the market is over.  But there is an enormous amount of idiosyncratic
      variation from subfield to subfield, person to person, and institution to
      institution, so the overall market conditions may not matter as much
      as you think, either for good or for ill.

    \end{quote}
  \item  \ifthenelse{\boolean{MyNotes}}{\marginpar{\tiny }}{} 
    How many job applications should I send out?

    \begin{quote}
      First, a key principle: You must not apply to a job that you know you would not
      accept under any circumstances, even if it were your only offer.  If
      at the end of the job market process you have at least one job offer
      and you refuse it, you forfeit the right to any further help from us
      in finding another job subsequently.  %(Of course, we reserve the right to relax this rule and choose to help you if you can persuade us that your reasons for turning the job down are compelling).

      Given that, there is a range of opinion on how many applications to send.  Some students basically apply to every employer advertising a job for which they could conceivably qualify, which at a maximum might lead to as many as 100 applications.  Others cull the list.  We \textit{strongly encourage} you to apply to a minimum of 25 potential employers.  Note that every extra employer you apply to puts a burden not just on you but on the staff, who must send out multiple reference letters for each employer.  So please do not apply to zero-probability places.  It's a waste of everybody's time.

      We have found that under these guidelines, most students apply to between 
      40 and 90 employers, though numbers can vary depending on circumstances 
      (e.g. a student with a Fulbright that requires them to apply only in 
      Europe may have a much more restricted set of potential choices,
      and therefore will apply to fewer employers than students who face
      no such restrictions).


      \ifdvi\phantomsection\hypertarget{FedHiringRules}{(FedHiringRules)}\fi

    \end{quote}
  \item  \ifthenelse{\boolean{MyNotes}}{\marginpar{\tiny }}{}  What are the Fed's rules about hiring persons who are not US citizens? 

    \begin{quote}
      For the Board of Governors, the answer is that it depends on the country you are from.  As of 2017, \href{https://www.federalreserve.gov/boarddocs/srletters/2006/SR0614a3.pdf}{the document linked here} which lists the eligible countries was found by Googling ``Does the Federal Reserve hire nonimmigrant aliens?'' 

      According to \href{https://www.federalreserve.gov/careers-faqs.htm}{this FAQ}, ``The Board will hire any qualified person who is a United States citizen or who is not a citizen but meets the requirements of the Immigration Reform and Control Act of 1986 (IRCA).  As required by law, the Board will not hire any person who is unable to satisfy the requirements identified in IRCA. In addition, citizens and nationals of the United States will receive preference in employment over equally qualified persons who are not citizens or nationals of the United States.''

      As I understand it, the Federal Reserve Banks can hire people from countries not in this list, but cannot give them ``Class 2'' clearance until they have served for two years, which means that the person cannot do much of what economists are hired at a Bank to do during that period.\footnote{For example, they cannot see the TealBook is the Fed's real-time assessment of economic conditions compiled before the FOMC meeting, and the preparation of which is a principal responsibility of economists at both the Board of Governors and at the Federal Reserve Banks.  It can never give them ``Class 1'' clearance, so such employees are limited in what they can do (which makes them much less attractive to the Fed).}  This means that the Banks only rarely hire persons from countries not on the list, and generally only when that person has some very unique skill or knowledge that cannot be obtained in any other way.

      As of the fall of 2017, it is my understanding that the Board and the Banks now sponsor people for Green cards after three years of employment.  (This seems to have changed over time, so if it is important to your job choice then you should verify it).

    \end{quote}
  \item  \ifthenelse{\boolean{MyNotes}}{\marginpar{\tiny }}{} 
    When I am ready for my recommendation letters to be sent (AFTER I have
    sent my applications to employers), what is the process?

    \begin{quote}
      See \url{https://llorracc.github.io/JobMarket/RecLetters} for detailed answers.


    \end{quote}
  \item   Can there be more references than people writing the recommendation 
    letters? For example, people who are not writing a letter but are willing to provide 
    an evaluation of the applicant in case the potential employer contacts them?

    \begin{quote}
      The ``References'' part of your CV is basically a list of people from
      whom the employers will expect to receive a letter.  For references who
      are not Hopkins faculty members, include complete contact information
      (address, phone number, email address).

      If you want to indicate that there are other people (not in your
      ``references'' list) who could be contacted, including their contact
      information somewhere else on the CV is a very obvious suggestion to
      the reader that the person is happy to be contacted (e.g.\ if you
      worked as an RA for someone who would be happy to recommend you,
      don't just list their name when describing the job, list their email
      address and phone number).


    \end{quote}
  \item  Do I need to write a different custom cover letter for every employer?
    
    \begin{quote}
      Yes - at least in the sense that you cannot write a generic ``Dear Potential Employer'' letter
      and photocopy it.  Mail-merge software needs to be used to make sure
      that every letter has the name of the chair of the relevant search
      committee, the institution and department to which you are applying,
      and the nature of the job for which you are applying (assistant
      professor, economist, etc).  The \Templates~ directory contains templates that 
      you can use to create the letters from your \EMtt~list.  
      
      You should do more highly customized letters for those employers
      where a) you have a special connection; b) you have special
      interests or a particularly close match to the employer's need; or
      c) there is some other reason to think they might read the letter
      and find it persuasive.  

      In most cases, it may not be worth your time to do more than
      5-10 customized letters.  (And, the \textit{first sentence} of the letter should signal
      clearly that it is customized -- people reading applications will probably assume that the letter is \textit{not} customized and will only glance at it briefly; if the first sentence looks like boilerplate, they may miss whatever customizations you have done later).

      \ifdvi\phantomsection\hypertarget{WhatIfEmployerContactsStudent}{(WhatIfEmployerContactsStudent)}\fi
    \end{quote}
  \item  What does it mean if an employer to whom I have not applied sends me an email asking me if I would be interested in applying?

    \begin{quote}
      This could mean any of several things.  One is that the employer has looked through the JHU candidates and has identified you as someone who they think might be an especially good fit (perhaps for reasons that are not obvious to you from their JOE posting, like the interests of existing faculty members).  Another is that they have heard something favorable about you from someone and want to get more information.  Yet another is that they might think that you ruled them out because you didn't know much about them but if you learned more about them you might find that they are better than you think.  (This is particularly true of schools that may have improved sharply in the recent past -- maybe as a result of an infusion of donor money, or hiring new faculty, or any of a host of other reasons, but whose improved quality has not yet been reflected in the (often badly out of date) rankings of departments on the internet).

      In any case, this is a signal of the school's interest in you, and generally it is a good idea to apply unless you have some strong reason not to do so (like, you would be ineligible to go there, or would not want to go there even if it were your only offer).  You can also seek your advisor(s) advice and that of the JMPO.


    \end{quote}
  \item  How does the video of the seminar work?
    
    \begin{quote}
      You should have identified a classmate (your `buddy') whose
      responsibility will be to make sure the video camera is recording
      and the video will be accessible to you during your seminar.  (The
      JMCC will provide info to everyone about how to do this).  

      Your `buddy' should also take detailed notes about the questions
      that come up during the seminar.  You should also ask all the other
      jobmarket candidates who attend your practice job seminar (and any
      friends you have who are not yet candidates) to take notes on how
      things go, and you should write up a summary of the combination of
      these notes (integrated with your own viewing of the video)
      afterwards.  This is the best way to be sure you absorb and process
      the feedback that you receive during the talk.  This is an
      exceptionally useful thing to do, and should probably take you at least
      a day; you should do it as soon as possible after the job talk, and
      make an appointment with your advisors the following week in which
      you share your writeup of the notes and sketch what you think you
      learned about how to improve.  Do NOT skimp on this; it is one of
      your best opportnities for making big improvements that you might
      otherwise never think of on your own.

    \end{quote}
  \item  Should I apply to schools who list jobs in fields that are not exactly my field but are related to my
    field?  Like, suppose my thesis is on international trade and a school lists
    an opening in ``international relations.''

    \begin{quote}
      You will have to excercise judgment on this.  For example, in the case
      above, if the ``International Relations'' job is in a business
      school's economics department and the job description mentions
      international trade, then yes, it is reasonable to apply.  If it's in
      a public policy school's International Relations department and your
      sense that they are actually looking to hire someone to bloviate about
      ``globalization'' then it's probably a waste of time.  Your advisers
      should be able to give you good advice in specific cases.

    \end{quote}
  \item  Some places ask for a statement of teaching and/or research
    interests.  Is this different than our 1 page
    dissertation summary?

    \begin{quote}
      A statement of teaching interests or teaching philosophy is
      definitely different from the dissertation summary.  There are some
      examples of previous students' statements in \Resources.  Many
      schools also ask for quantitative information about teaching
      evaluations.  We have provided a template for that too in
      \Templates.  Please provide feedback to the JMCC or JMPO if you
      think this template could be usefully modified -- it is fairly new.

      If the school asks for a statement of research interests it is probably
      a good idea to include such a statement separately from the dissertation
      abstract, though in some cases (use your judgement) the abstract will suffice.  Again, some examples are in \Resources.  
      The key thing you can communicate here is a sense of where your longer-term 
      interests are -- what do you anticipate working on next once your dissertation
      papers are published?  How does your work fit into broader developments in 
      the economic research literature?

      \ifdvi\phantomsection\hypertarget{PostOtherDocsQ}{(PostOtherDocsQ)}\fi
    \end{quote}
  \item  The {\JOE} posting for many jobs asks for only a subset of all the materials I have assembled.  Should I upload everything any as ``other documents''?

    \begin{quote}
    No. The employers in question probably are trying to cut down on the massive amount of bandwidth/clutter they get from the process.  They have probably chosen what to ask for carefully.  There are only two kinds of content that you should upload: If they don't ask for your job market paper, you should upload it. And if you have any one thing that is especially notable that they might be particularly impressed or intrigued by (your sole-authored AER paper, say!) you should upload that.

  \end{quote}
\item  Will the department favor me over my classmate doing similar work?  Alternatively, how do I crush/avoid crushing my long-time friend in the job market?

  \begin{quote}
    This is a (much) less important issue than you might think, because even when schools say they are looking in ``all fields'' they usually are actually strongly biased toward filling some particular need.  Even if they say they are looking in ``health'' they may be really looking for an IO person in health, or really looking for a structural applied micro health person, or a public finance health person, or whatever.  Even at big institutions (the Fed; the IMF) the positions are available for particular groups doing particular things.  So in practice, there is very little real issue of ``student rivalry'' in the sense of worrying about whether other JHU students are your competitors for a given job, or whether the placement director might tilt the playing field somehow; in my (many) years as a placement director I have never seen a case where two JHU students were actually close substitutes for a given match.  As a result, my incentives are almost perfectly aligned with the students' incentives, in the sense that my job performance is measured by how well JHU students do on the market, and my incentive is to help find the right matches.

  \end{quote}

\item  It is fairly early (November or early December), and I have received several requests for interviews to my ``backup'' places but might want to cancel them if I have enough interviews at my ``preferred'' places later.  Is it OK to accept interviews at my ``backup'' places for now and then back out later if it turns out that I need to use the slots for better places?

  \begin{quote}
    Yes, you can back out later.  On the whole it is not wise to do so unless every moment of your possible interview time is filled and then someone calls who is preferred to some of your existing interviews.  Again, remember the principle that you are likely to be interacting with many of the same people throughout your career, and you don't want your first impression on them to be a bad one.  Also, you may ultimately only receive a job offer from your ``backup'' places, not the ones you hope for!

  \end{quote}
  \ifdvi\phantomsection\hypertarget{InterviewBestTime}{(InterviewBestTime)}\fi
  
\item  I've received an interview request for one of my most preferred employers. When is the best time to pick to do the interview?
  \begin{itemize}
  \item Morning, but not the first slot -- you and the interviewers will not yet be tired
  \item The second day of not the first day of the conference
    \begin{itemize}
    \item You'll have had a few practice interviews to get in the groove, and will still be ``fresh'' and not robotic sounding
    \item The interviewers similarly will have had enough interviews to be in their groove, but will also still be fresh
    \end{itemize}
  \end{itemize}

\item  Should I ask whether my expenses will be reimbursed by schools that fly me out for an interview?
  \begin{quote}
    No.  Expenses are always reimbursed (if you keep good records and submit an expense report).  For this purpose, \url{http://expensify.com} is useful.

  \end{quote}

\item  What happens during a flyout?
  \begin{quote}
    Obviously, you will give your job market seminar.  But more is
    involved.  You generally meet individually with faculty at a
    fly-out.  Usually a few before your talk, then give your talk, then
    lunch, then meet with more faculty, then dinner.  Somewhere in there
    you may also meet with the Dean, so you might want to be prepared
    with a question or two for him/her (one could be something about
    their tenure policy).  Occasionally, at big schools, you might meet
    with groups of a few faculty (so that everyone has a chance to meet
    you).  The conversations are generally more relaxed and informal
    than the interviews at the meetings, and also more wide-ranging.
    This is where it really pays off to be an avid reader of the \textit{Economist},
    the \textit{Financial Times}, good economics blogs, and other news sources.
    If someone mentions that they read a recent story related to your 
    work in the \textit{Economist} and you haven't read it, you'll look like
    you're not really on top of your field.  If your conversation strays
    a long way from your dissertation but the faculty member is impressed 
    by your general knowledge, that's a very good thing.

    These meetings often also focus on where your research going from
    here, and maybe some discussion about teaching experience (especially
    at small liberal arts colleges).  It's impossible to prepare for
    everyone you're going to meet (especially since you usually don't get
    the schedule until late the evening before).  But in the meetings, you
    can also ask some of them what they are working on, how they like the
    school, what their experience has been with undergrad RAs, etc.  They
    also want to make sure that they would like you as a colleague.  (It is
    also a really long day, not that there's anything to do about that other
    than getting enough sleep the night before.)



  \end{quote}

\item  \ifthenelse{\boolean{MyNotes}}{\marginpar{\tiny }}{} 
  What happens if I don't get any job offers?
  \begin{quote}
    First, don't start worrying too early.  The way the job market works
    is as follows: The market decides who are the stars for the year, and
    those people each get 25 offers.  The 22 places that have no chance of
    hiring them (along with the 3 who do have a chance) then wait for a
    month until they learn the candidate has turned them down.  They then
    make an offer to one of the other stars, who has already accepted a
    job at Harvard and also turns them down.  Then, in March, they get
    realistic and start making offers to people they might actually be
    able to hire.  So if you don't have any job offers by Feb 15, don't
    worry.  (But do keep in touch with your advisors and with me during this
    period).

    Usually the candidates who have the most difficulty finding jobs are
    those with strong restrictions (usually geographical).  In other
    words, if you say ``I can only take a job in New Jersey'' then it can
    be very difficult - this is the only category of student who 
    have sometimes not been able to find a job in the last few years.

    As noted above, if things don't work out on the market this year, then
    you have the right to our help again next year.

  \end{quote}

\item  Does the department pay any of my expenses for the job market?

  \begin{quote}  
  At some schools, students going on the market are charged a fee to cover the costs of administrative assistance in the department's marketing of their students.  We provide the services of the JMPO, the JMCC, the job market staff person, the web liaison, and faculty who can help you in various other ways.  But that is the extent of our ability to help with the expenses associated with the market.  Think of your expenses as an investment!


\end{quote}

\item  How long should my job market paper be?

  \begin{quote}
    Maximum of 35 double-spaced pages (excluding figs, tables,
    appendices), min 12 point type, 1.25 inch margins.  If figs and
    tables are included in the text, maximum of 42 pages.   
    
    A wise person once said that a paper is not finished when nothing
    remains to be added; it is finished when nothing remains that could
    be taken away.
    
    If you think you can't fit everything into these guidelines, think
    again.  You've got to realize that \textit{nobody is going to read a
      huge fat paper just because you spent a lot of time writing it}.
    Get to the point.  No rambling pages of text designed to prove you've
    read every paper since the seminal contribution of Moses (1200 BC).

    There is one escape clause to the above: If necessary, you can have
    substantial online appendices containing tables, proofs, and other material, and
    refer to them briefly in the text.  This is an excellent way to
    establish points that may be necessary to the argument of your paper
    but not particularly interesting in themselves.   

  \end{quote}

\item  How long should the cover letter be?
  \begin{quote}
    One page.  Period.

  \end{quote}

\item  What should be in the abstract?
  \begin{quote}
    It can either be an abstract for your job market paper or for the dissertation 
    as a whole.  It must be one page, but does not need to fill the page.  Min 12 point
    type, 1.25 inch margins.  Don't try to squeeze the maximum possible number of words onto
    the page.  It is more important that it be readable and look professional than that 
    it mention everything you have done.

  \end{quote}

\item  What's to stop me from applying for a job in JOE even if my advisors don't think I'm ready?  I'll just send my application and hope that someone will hire me even without advisors' approval.  (Nobody has been bold enough to ask this question directly, but I'm sure students have thought about it).

  \begin{quote}
    There is no point in applying to employers unless you have approval from advisors and a reasonable job market paper.  Every job posting has a large number of applications.  The first thing every employer does is to look at the letters.  If there are no letters, or the letters basically say ``this candidate is not prepared,'' the employer will throw your application in the trash and look at the next applicant in the pile of 400 other people who applied for the job.  If your letters are adequate, the employer looks next at the job market paper.  If it is incoherent, the employer A) throws your application away, and B) mentally downgrades JHU for the purpose of future hiring.  Hence our absolute requirement that you have a coherent job market paper or you're not going on the market.

    Just to make it crystal clear, I will say it again: If your advisors
    are not prepared to write a letter that says you should be hired, no
    employer will hire you.  So there is no point in trying to apply on
    your own without your advisors' approval - it would be a waste of
    paper, money, and time on your part.  Your application will simply be
    ignored if it does not come with advisors' letters, and will be discarded
    if it comes with letters that say you are not ready.

    (Of course, I am referring here to jobs which are part of the usual PhD hiring 
    process in economics.  There's no reason you can't search for another kind of 
    job, say as a brain surgeon or an astronaut, which does not require a 
    PhD in economics.  And of course you are welcome to apply to jobs that require 
    only a mains degree in economics, for which you can reasonably request letters 
    from anyone who has supervised you in the JHU program).  

  \end{quote}

\item  After the interviews at the ASSA meetings, should I send a ``thank you'' email to the interviewers, or is that too pushy/annoying?
  \begin{quote}
    It is best NOT to send a generic email ``Dear x, Thanks for interviewing me at the ASSA meetings.''  But it's OK, and maybe even a good idea, to send a BRIEF and SPECIFICALLY TAILORED message to specific interviews at a selection of the places you are most interested in.  ``Professor Putin, Thank you for the suggestion that I look into the mortgage market of Burkina Faso because its institutional details provide a test of my theory.''

    \hypertarget{restricted-data}{}  
  \end{quote}

\item  My job market paper uses restricted data, and I must obtain approval from the provider of the data before posting it. But the deadline for posting job market materials has arrived. What should I do?

  \begin{quote}
    First point: You should submit the version of the paper you want approved LONG ENOUGH BEFOREHAND to have an approved version by the time we need to post the paper (early October).

    If you have not received approval by then:
    
      Options:
      \begin{enumerate}
      \item Make a PDF document whose entire content is one page saying ``This paper uses restricted data and cannot be posted publicly until [whoever] has approved its release.''
      \item Make a PDF document that contains the first page of your JMP (including abstract) and the second page can read as in 1.
      \end{enumerate}

    Option 2.\ (or a modified version that redacts any ``results'' from the abstract) is much better. If you make such abstract redactions, you should see whether you can get the provider of the data to quickly review either the redacted or the full (unredacted) abstract, faster than they may be able to review the whole paper.

  \end{quote}
\end{enumerate}

\end{document}
